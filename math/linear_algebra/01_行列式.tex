\documentclass{article}
\usepackage{fontspec}
\usepackage{graphicx}
\usepackage{type1cm}
\usepackage{geometry}
\geometry{a4paper,left=1cm,right=1cm,top=3cm,bottom=3cm}
\usepackage[bold-style=ISO]{unicode-math}
% 添加heading=true, 使用中文版式
\usepackage[heading=true]{ctex}
% 生成pdf书签
\usepackage{hyperref}
% 自定义多级标题格式的宏包
\usepackage{titlesec} 
\titleformat{\section}[block]{\Huge\bfseries}{\arabic{section}}{1em}{}[]
\titleformat{\subsection}[block]{\huge\bfseries}{\arabic{section}.\arabic{subsection}}{1em}{}[]
\titleformat{\paragraph}[block]{\LARGE\bfseries}{[\arabic{paragraph}]}{1em}{}[]

\begin{document}
\begin{flushleft}
	\LARGE
	
	%左斜三点
	\newcommand{\udots}{\mathinner{\mskip1mu\raise1pt\vbox{\kern7pt\hbox{.}}  
			\mskip2mu\raise4pt\hbox{.}\mskip2mu\raise7pt\hbox{.}\mskip1mu}} 


	\section{逆序数}
	在一个排列中,如果前面的数大于后面的数,那么它们就称为一个逆序。\\
	一个排列中逆序的总数就称为这个排列的逆序数。记为$\tau(i_1,i_2,...,i_n)$\\
	例:求$\tau(32514)$\\
	\qquad 3后面比它小的数有2个\\
	\qquad 2后面比它小的数有1个\\
	\qquad 5后面比它小的数有2个\\
	\qquad 1后面比它小的数有0个\\
	\qquad 4后面比它小的数有0个\\
	\qquad 即$\tau(32514)=2+1+2+0+0=5$\\
	
	\section{行列式计算}
	\paragraph{二阶行列式}
	$\left|\begin{array}{cccc} 
		a_{11}&a_{12}\\ 
		a_{21}&a_{22}
	\end{array}\right|=a_{11}a_{22}-a_{12}a_{21}$\\

	\paragraph{三阶行列式}
	$\left|\begin{array}{cccc} 
		a_{11}&a_{12}&a_{13}\\ 
		a_{21}&a_{22}&a_{23}\\
		a_{31}&a_{32}&a_{33}
	\end{array}\right|$\\

	
	
	\section{行列式的性质}
	1、行列互换,行列式的值不变\\
	2、两行(列)互换,行列式的值变号\\
	\qquad 推论:两行(列)相同,行列式的值为0\\
	3、可以把某行(列)的公因子k提到行列式的外面\\
	\qquad 推论:某行(列)为0,行列式的值为0\\
	\qquad 推论:某两行(列)的元素对应成比例,行列式的值为0\\
	4、$\left|\begin{array}{cccc} 
	a_{11}+b&a_{12}+c&a_{13}+d\\ 
	a_{21}&a_{22}&a_{23}\\
	a_{31}&a_{32}&a_{33}
	\end{array}\right|=\left|\begin{array}{cccc} 
	a_{11}&a_{12}&a_{13}\\ 
	a_{21}&a_{22}&a_{23}\\
	a_{31}&a_{32}&a_{33}
	\end{array}\right|+\left|\begin{array}{cccc} 
	b&c&d\\ 
	a_{21}&a_{22}&a_{23}\\
	a_{31}&a_{32}&a_{33}
	\end{array}\right|$\\
	5、某行(列)的k倍加到另外一行(列)上,行列式的值不变\\
	
	\section{行列式按行(列)展开}
	
	\subsection{余子式}
	余子式$M_{ij}$为去掉第$i$行和第$j$列后,剩下的元素组成的行列式\\
	$\left|\begin{array}{cccc} 
	a_{11}&a_{12}&a_{13}\\ 
	a_{21}&a_{22}&a_{23}\\
	a_{31}&a_{32}&a_{33}
	\end{array}\right|$的一个余子式$M_{22}=\left|\begin{array}{cccc} 
	a_{11}&a_{13}\\
	a_{31}&a_{33}
	\end{array}\right|$\\
	
	\subsection{代数余子式}
	$A_{ij}=(-1)^{i+j}M_{ij}$\\
	
	\subsection{行列式按行(列)展开公式}
	行列式的值等于它的任意一行元素与其对应的代数余子式乘积之和
	$|A|=a_{i1}A_{i1}+a_{i2}A_{i2}+...+a_{in}A_{in}$\\
	
	\section{重要行列式}
	
	\subsection{上(下)三角}
	\paragraph{主对角线}
	$\left|\begin{array}{cccc} 
		a_{11}&a_{12}&\cdots&a_{1n}\\ 
		0&a_{22}&\cdots&a_{2n}\\ 
		\vdots&\vdots&\ddots&\vdots\\ 
		0&0&\cdots&a_{nn}\\ 
	\end{array}\right|=a_{11}a_{22}\cdots a_{nn}$\\
	~\\
	$\left|\begin{array}{cccc} 
		a_{11}&0&\cdots&0\\ 
		a_{21}&a_{22}&\cdots&0\\ 
		\vdots&\vdots&\ddots&\vdots\\ 
		a_{n1}&a_{n2}&\cdots&a_{nn}\\ 
	\end{array}\right|=a_{11}a_{22}\cdots a_{nn}$\\

	\paragraph{副对角线}
	$\left|\begin{array}{cccc} 
		a_{11}&\cdots&a_{1,n-1}&a_{1n}\\ 
		a_{21}&\cdots&a_{2,n-1}&0\\ 
		\vdots&\udots&\vdots&\vdots\\ 
		a_{n1}&\cdots&0&0\\ 
	\end{array}\right|=(-1)^{\frac{n(n-1)}{2}}a_{1n}a_{2,n-1}\cdots a_{n1}$\\
	~\\
	$\left|\begin{array}{cccc} 
		0&\cdots&0&a_{1n}\\ 
		0&\cdots&a_{2,n-1}&a_{2n}\\ 
		\vdots&\udots&\vdots&\vdots\\ 
		a_{n1}&\cdots&a_{n,n-1}&a_{nn}\\ 
	\end{array}\right|=(-1)^{\frac{n(n-1)}{2}}a_{1n}a_{2,n-1}\cdots a_{n1}$\\

	\subsection{拉普拉斯展开式}
	
	
\end{flushleft}
\end{document}