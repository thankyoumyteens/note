\documentclass{article}
\usepackage{fontspec}
\usepackage{graphicx}
\usepackage{type1cm}
\usepackage{geometry}
\usepackage[bold-style=ISO]{unicode-math}
\usepackage[heading=true]{ctex}%添加heading=true,使用中文版式
\geometry{a4paper,left=1cm,right=1cm,top=3cm,bottom=3cm}
\usepackage{titlesec} %自定义多级标题格式的宏包
\titleformat{\section}[block]{\Huge\bfseries}{\arabic{section}}{1em}{}[]
\titleformat{\subsection}[block]{\huge\bfseries}{\arabic{section}.\arabic{subsection}}{1em}{}[]
\titleformat{\paragraph}[block]{\LARGE\bfseries}{[\arabic{paragraph}]}{1em}{}[]

\begin{document}
\begin{flushleft}
	\LARGE
	
	\section{导数}
	
	可导必连续,不连续必不可导\\
	~\\
	求某一点$x_0$的导数:用导数的定义$\lim\limits_{x\to x_0} \frac{f(x)-f(x_0
		)}{x-x_0}$\\
	~\\
	导数的几何意义:切线的斜率$k$\\
	切线方程:$y-y_0=k(x-x_0)$\\
	~\\
	设$f(x)$是$(-a,a)$上的偶(奇)函数且可导,则$f(x)'$是$(-a,a)$上的奇(偶)函数\\
	设$f(x)$以$T$为周期且可导,则$f(x)'$也以$T$为周期\\
	~\\
	可导$\pm$不可导$=$不可导\\
	~\\
	$y=f(x)$的反函数为$x=\phi(y)$,则$f'(x)=\frac{1}{\phi'(y)}$\\
	~\\
	复合函数求导:从外到里层层求导\\
	~\\
	$f(x)$可导,$g(x)$连续但不可导,若$f(x)\cdot g(x)$在$x_0$处可导,则$f(x_0)=0$\\
	
	\section{基本求导公式}
	
	$(C)'=0$\\
	$(x^u)'=ux^{u-1}$\\
	$(\sin x)'=\cos x$\\
	$(\cos x)'=-\sin x$\\
	$(\tan x)'=\sec^2 x$\\
	$(\cot x)'=-\csc^2 x$\\
	$(\sec x)'=\sec x\tan x$\\
	$(\csc x)'=-\csc x\cot x$\\
	$(a^x)'=a^xlna$\\
	$(e^x)'=e^x$\\
	$(log_ax)'=\frac{1}{xlna}$\\
	$(lnx)'=\frac{1}{x}$\\
	$(\arcsin x)'=\frac{1}{\sqrt{1-x^2}}$\\
	$(\arccos x)'=-\frac{1}{\sqrt{1-x^2}}$\\
	$(\arctan x)'=\frac{1}{1+x^2}$\\
	$(arccot x)'=-\frac{1}{1+x^2}$\\
	$(ln|x|)'=\frac{1}{x}$\\
	
	\section{导数的四则运算}
	
	设$u,v$均可导,则:\\
	$(u\pm v)'=u'+v'$\\
	$(uv)'=u'v+uv'$\\
	$(\frac{u}{v})'=\frac{u'v-uv'}{v^2}$\\
	
	\section{求n阶导数}
	
	\subsection{常用公式}
	
	1、$(e^{ax+b})^{(n)}=a^ne^{ax+b}$\\
	2、$[\sin(ax+b)]^{(n)}=a^n\sin(ax+b+\frac{n\pi}{2})$\\
	3、$[\cos(ax+b)]^{(n)}=a^n\cos(ax+b+\frac{n\pi}{2})$\\
	4、$[ln(ax+b)]^{(n)}=(-1)^{n-1}a^n\frac{(n-1)!}{(ax+b)^n}$\\
	5、$(\frac{1}{ax+b})^{(n)}=(-1)^{n}a^n\frac{(n)!}{(ax+b)^{n+1}}$\\
	~\\
	若$f(x)$可分解成$f_1(x)$和$f_2(x)$,且知道$f_1^{(n)}(x)$和$f_2^{(n)}(x)$,则$f^{(n)}(x)=f_1^{(n)}(x)+f_2^{(n)}(x)$\\

	\subsection{莱布尼兹公式}
	
	$(uv)^{(n)} = \sum\limits_{k=0}^{n} C_n^k u^{(n-k)} v^{(k)} = \sum\limits_{k=0}^{n} C_n^k u^{(k)} v^{(n-k)}$\\
	$C_n^m=\frac{n!}{m!(n-m)!}$\\
	$C_n^0=1$\\
	
	\section{隐函数求导}
	
	$y=y(x)$由方程$F(x,y)=0$确定,\\
	在方程$F(x,y)=0$两端直接对$x$求导\\
	
	\section{参数方程求导}
	
	$y=y(x)$由参数方程$\left\{
	\begin{array}{rcl}
	x=\phi(t)\\
	y=\Phi(t)
	\end{array} \right.$确定,且$\phi(t)$和$\Phi(t)$均可导,则$y'=\frac{y'_t}{x'_t}=\frac{\Phi'(t)}{\phi'(t)}$\\
	
	\section{微分}
	
	微分的定义:设函数$y=f(x)$在$x_0$的邻域内有定义,如果函数的增量$\Delta y = f(x_0 + \Delta x) − f(x_0)$可表示为$ \Delta y = A\Delta x + o(\Delta x)$(其中$A$是不依赖于$\Delta x$的常数,那么称函数$f(x)$在点$x_0$是可微的,且$A\Delta x$称作函数在点$x_0$相应于自变量增量$\Delta x$的微分,记作$dy$,即$dy = A\Delta x$\\
	若$x$是自变量,则$\Delta x=dx$\\
	~\\
	可微必可导,可导必可微,且$dy=f'(x)dx$\\
	~\\
	\includegraphics[scale=1.0]{1.jpg}
	
\end{flushleft}
\end{document}