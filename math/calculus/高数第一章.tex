\documentclass{article}
\usepackage{fontspec}
\usepackage{type1cm}
\usepackage{geometry}
\usepackage[bold-style=ISO]{unicode-math}
\usepackage[heading=true]{ctex}%添加heading=true,使用中文版式
\geometry{a4paper,left=1cm,right=1cm,top=3cm,bottom=3cm}
\usepackage{titlesec} %自定义多级标题格式的宏包
\titleformat{\section}[block]{\Huge\bfseries}{\arabic{section}}{1em}{}[]
\titleformat{\subsection}[block]{\huge\bfseries}{\arabic{section}.\arabic{subsection}}{1em}{}[]
\titleformat{\paragraph}[block]{\huge\bfseries}{[\arabic{paragraph}]}{1em}{}[]

\begin{document}
\begin{flushleft}
	\huge
	
	\section{公式}
	\subsection{函数}
	
	\paragraph{取整函数}
	$y=[x]$ 向左取整: $x-1<[x]\leq x$\\
	一般搭配夹逼准则\\
	\paragraph{幂指函数}
	$y=f(x)^{g(x)}$\\
	~\\
	定义域关于原点对称的函数,一定能写成奇函数+偶函数的形式\\
	应用-积分公式:\\
	$\int_{-a}^a f(x)dx=\int_0^a [f(x)+f(-x)]dx$\\
	
	\subsection{极限}
	
	\paragraph{要分左右极限的情况}
	1.分段函数的分段点处\\
	2.e的无穷大型,如$\lim\limits_{x\to 1} e^{\frac{1}{x-1}}$\\
	3.$arctan\infty$型,如$\arctan{\frac{1}{x-1}}$\\
	~\\
	\paragraph{极限的四则运算}
	设$\lim f(x)=A,\lim g(x)=B$,则:\\
	1.$\lim [f(x)\pm g(x)]=A\pm B$\\
	2.$\lim [f(x)g(x)]=AB$\\
	3.$\lim \frac{f(x)}{g(x)} =\frac{A}{B},(B\neq 0)$\\
	~\\
	若$\lim f(x)$存在$\lim g(x)$不存在,则$\lim [f(x)\pm g(x)]$不存在\\
	其他情况都没有结论\\
	~\\
	\paragraph{$\frac{\mbox{多项式}}{\mbox{多项式}}$求极限}
	$\lim\limits_{x\to \infty} \frac{a_0x^m+...+a_mx^0}{b_0x^n+...+b_nx^0}=
	\left\{
	\begin{array}{rcl}
	\frac{a_0}{b_0}, & & {m=n}\\
	0, & & {m<n}\\
	\infty, & & {m>n}
	\end{array} \right.$\\
	例:\\
	$\lim\limits_{x\to \infty} \frac{3x^3+4x^2+2}{7x^3+5x^2-3} = \frac{3}{7}$\\
	~\\
	\paragraph{复合函数求极限}
	如果$f(x)$连续,且$g(x)$有极限A,则:\\
	$\lim\limits_{x\to x_0} f[g(x)]=f[\lim\limits_{x\to x_0}g(x)]=f(A)$\\
	例:\\
	$\lim\limits_{x\to 3} \sqrt{\frac{x-3}{x^2-9}}= \sqrt{\lim\limits_{x\to 3} \frac{x-3}{x^2-9}}=\sqrt{\frac{1}{6}}$\\
	~\\
	\paragraph{幂指函数求极限}
	若$\lim f(x)=A>0$且$\lim g(x)=B$,则:$\lim f(x)^{g(x)}=A^B$\\
	
	\subsection{重要极限}
	
	\paragraph{夹逼准则}
	函数$A>B>C$, 函数$A$的极限是$X$,函数$C$的极限也是$X$,那么函数$B$的极限就一定是$X$\\
	\paragraph{单调有界准则}
	单调递增且有上界,则有极限, 单调递减且有下界,则有极限\\
	~\\
	\paragraph{重要极限}
	$\lim\limits_{x\to 0} \frac{\sin x}{x}=1$\\
	$\lim\limits_{x\to 0} \frac{\tan x}{x}=1$\\
	$\lim\limits_{x\to 0} \frac{\arcsin x}{x}=1$\\
	$\lim\limits_{x\to \infty} (1+\frac{1}{x})^x=e$\\
	$\lim\limits_{x\to 0} (1+x)^{\frac{1}{x}}=e$\\
	
	\subsection{无穷小}
	
	无穷小:极限为0,(0也是无穷小)\\
	有界函数$\times$无穷小 仍是无穷小\\
	~\\
	设$\alpha$和$\beta$是无穷小,且$\alpha \neq 0$,
	若$\lim \frac{\beta}{\alpha}=0$,则$\beta$是比$\alpha$的高阶无穷小,
	记为:$\beta = o(\alpha)$\\
	~\\
	$o(x^2)\pm o(x^2)=o(x^2)$\\
	$o(x^2)\pm o(x^3)=o(x^2)$\\
	$x^2 o(x^3)=o(x^5)$\\
	$o(x^2) o(x^3)=o(x^5)$\\
	$o(2x^2)=o(x^2)$\\
	
	\subsection{常用的等价}
	
	若$\lim \frac{\beta}{\alpha}=1$,则$\beta$与$\alpha$是等价无穷小,
	记为:$\beta \sim \alpha$\\
	~\\
	$\beta \sim \alpha \Leftrightarrow \beta = \alpha + o(\alpha)$\\
	$x$的高次方$\pm x$的低次方$\sim x$的低次方\\
	\ \ 例:$x^3+3x\sim 3x$\\
	~\\
	若$\alpha \sim \alpha_1$且$\beta \sim \beta_1$,则$\lim \frac{\beta}{\alpha} = \lim \frac{\beta_1}{\alpha_1}$\\
	~\\
	当$x\to 0$时,$\sin x \sim x$\\
	当$x\to 0$时,$\arcsin x \sim x$\\
	当$x\to 0$时,$\tan x \sim x$\\
	当$x\to 0$时,$\arctan x \sim x$\\
	~\\
	当$x\to 0$时,$ln(1+x) \sim x$\\
	当$x\to 0$时,$e^x-1 \sim x$\\
	~\\
	当$x\to 0$时,$1-\cos x \sim \frac{1}{2} x^2$\\
	当$x\to 0$时,$\sec x - 1 \sim \frac{1}{2} x^2$\\
	~\\
	当$x\to 0$时,$(1+\alpha x)^\beta -1 \sim \alpha\beta x$\\
	~\\
	当$x\to 0$时,$\alpha^x -1 \sim xln\alpha$\\
	
	\subsection{连续}
	
	若$f(x)$在$x_0$处连续,则$\lim\limits_{x\to x_0} f(x)=f(x_0)$\\
	~\\
	连续$\pm\times\div$连续$=$连续\\
	连续$\pm$不连续$=$不连续\\
	若$f(x)$连续$g(x)$也连续,则$f[g(x)]$连续\\
	~\\
	单调连续函数的反函数也连续,且单调性相同\\
	~\\
	闭区间内连续函数必有界\\
	推广:\\
	$f(x)$在$(a,b)$内连续,且$\lim\limits_{x\to a^+} f(x)$和$\lim\limits_{x\to b^-} f(x)$都存在,则$f(x)$在$(a,b)$内有界\\
	~\\
	\paragraph{零点定理}
	$f(x)$在$(a,b)$内连续,且$\lim\limits_{x\to a^+} f(x)$和$\lim\limits_{x\to b^-} f(x)$异号,则$\exists \xi \in (a,b)$,使得$f(\xi)=0$\\
	
	\subsection{间断点}
	
	\paragraph{第一类间断点}
	1.可去间断点:左右极限均存在且相等\\
	2.跳跃间断点:左右极限均存在且不相等\\
	\paragraph{第二类间断点}
	左右极限至少一个不存在\\
	1.无穷间断点:$x\to x_0^-$或$x\to x_0^+$时,$f(x)\to \infty$\\
	2.振荡间断点:$x\to x_0^-$或$x\to x_0^+$时,$f(x)$上下振荡\\

\end{flushleft}
\end{document}