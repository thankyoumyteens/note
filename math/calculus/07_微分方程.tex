\documentclass{article}
\usepackage{fontspec}
\usepackage{type1cm}
\usepackage{geometry}
\usepackage[bold-style=ISO]{unicode-math}
\usepackage[heading=true]{ctex}%添加heading=true,使用中文版式
\geometry{a4paper,left=1cm,right=1cm,top=3cm,bottom=3cm}
\usepackage{titlesec} %自定义多级标题格式的宏包
\titleformat{\section}[block]{\Huge\bfseries}{\arabic{section}}{1em}{}[]
\titleformat{\subsection}[block]{\huge\bfseries}{\arabic{section}.\arabic{subsection}}{1em}{}[]
\titleformat{\paragraph}[block]{\LARGE\bfseries}{[\arabic{paragraph}]}{1em}{}[]

\begin{document}
\begin{flushleft}
	\LARGE
	% \qquad 缩进两个字符
	
	\section{公式}
	
	\subsection{微分方程的基本概念}
	
	\paragraph{微分方程}\qquad 
	表示未知函数,未知函数的导数与自变量之间的关系的方程\\
	\paragraph{微分方程的阶}
	未知函数最高阶导数的阶数\\
	\paragraph{微分方程的解}
	找到一个函数代入微分方程能使该方程成为恒等式,找到的这个函数就是微分方程的解\\
	\paragraph{微分方程的通解}
	找到的微分方程中含有任意常数,即我们经常用C表示常数,这一类函数能使微分方程成为恒等式,统称为微分方程的通解\\
	\paragraph{微分方程的特解}
	在微分方程的基础上给出了初始条件(通常给出x和y的值关系)来确定出那个常数C,从而确定出一个函数,这个函数即为该微分方程的特解\\
	
	\subsection{一阶微分方程}
	
	 \paragraph{可分离变量的微分方程}
	 定义:形如$g(y)dy=f(x)dx$的方程\\
	 解法:方程两端直接积分,得出$G(y)=F(x)+C$\\
	 
	 \paragraph{齐次方程}
	 定义:形如$\frac{dy}{dx}=\phi(\frac{y}{x})$的方程\\
	 解法:令$u=\frac{y}{x}$做换元,化成可分离变量的微分方程\\
	 
	 \paragraph{一阶线性微分方程}
	 定义:形如$\frac{dy}{dx}+P(x)y=Q(x)$的方程\\
	 当$Q(x)=0$时,称为一阶线性齐次微分方程\\
	 当$Q(x)\neq 0$时,称为一阶线性非齐次微分方程\\
	 解法:$y=e^{-\int P(x)dx}(\int Q(x)e^{\int P(x)dx}dx+C)$\\
	 
	 \paragraph{伯努利方程}
	 定义:形如$\frac{dy}{dx}+P(x)y=Q(x)y^n$的方程\\
	 解法:方程两端同乘$y^{-n}$,再令$z=y^{1-n}$做换元,化成一阶线性微分方程\\
	 
	 \subsection{高阶微分方程求解}
	 
	 \paragraph{高阶可降价微分方程}
	 1、$y^{(n)}=f(x)$型\\
	 解法:方程两端直接做n次积分\\
	 ~\\
	 2、$y''=f(x,y')$型\\
	 解法:令$y'=p$,则$y''=p'$,带入原方程,化为一阶方程\\
	 ~\\
	 3、$y''=f(y,y')$型\\
	 解法:令$y'=p$,则$y''=p\frac{dp}{dy}$,带入原方程,化为一阶方程\\

	 \paragraph{高阶线性微分方程}
	 1、设$y_1$和$y_2$是$y''+P(x)y'+Q(x)y=0$的两个无关解(不成比例的解),则$Y=C_1y_1+C_2y_2$是该齐次方程的通解,其中$C_1$和$C_2$是任意常数\\
	 2、设$y^*$是$y''+P(x)y'+Q(x)y=f(x)$的一个特解,$Y$是对应的齐次方程的通解,则$y=Y+y^*$是该非齐次方程的通解\\
	 3、设$y''+P(x)y'+Q(x)y=f_1(x)+f_2(x)$,其中$y_1^*$是对应$f_1(x)$的特解,$y_2^*$是对应$f_2(x)$的特解,则$y_1^*+y_2^*$是该非齐次方程的特解\\
	 4、设$y_1^*$和$y_2^*$都是$y''+P(x)y'+Q(x)y=f(x)$的特解,则$y_1^*-y_2^*$是对应的齐次方程的解\\
	 
	 \paragraph{二阶常系数齐次线性微分方程}
	 定义:形如$y''+py'+qy=0$的方程\\
	 解法:写出特征方程$r^2_pr+q=0$\\
	 1、若特征方程有两个不相等实根$r_1\neq r_2$,\\
	 则通解为$y=C_1e^{r_1x}+C_2e^{r_2x}$\\
	 2、若特征方程有两个相等实根$r_1=r_2$,\\
	 则通解为$y=(C_1+C_2x)e^{r_1x}$\\
	 3、若特征方程有两个虚根$r_{1,2}=\alpha\pm\beta_i$,\\
	 则通解为$y=e^{\alpha x}(C_1cos\beta x+C_2sin\beta x)$\\
	 
	 \paragraph{二阶常系数非齐次线性微分方程}
	 定义:形如$y''+py'+qy=f(x)$的方程\\
	 1、若$f(x)=e^{\lambda x}P_m(x)$,则特解为$y^*=e^{\lambda x}R_m(x)x^k$\\
	 其中$R_m(x)$是$m$次一般多项式\\
	 $k=\left\{
	 \begin{array}{rcl}
	 0,& & \lambda\mbox{不是特征方程的根}\\
	 1,& & \lambda\mbox{是特征方程的单根}\\
	 2,& & \lambda\mbox{是特征方程的重根}
	 \end{array} \right.$\\
	 2、若$f(x)=e^{\lambda x}[P_m(x)cos\omega x+Q_n(x)sin\omega x]$,则特解为$y^*=e^{\lambda x}[R_l(x)cos\omega x+S_l(x)sin\omega x]x^k$\\
	 其中$R_l(x)$和$S_l(x)$是两个不同的$m$次一般多项式,且$l=max(m,n)$\\
	 $k=\left\{
	 \begin{array}{rcl}
	 0,& & \lambda\pm\omega i\mbox{不是特征根}\\
	 1,& & \lambda\pm\omega i\mbox{是特征根}
	 \end{array} \right.$\\
	 
	 \paragraph{二阶欧拉方程}
	 定义:形如$x^2y''+axy'+by=f(x)$的方程\\
	 解法:令$x=e^t (x>0)$或$x=-e^t (x<0)$换元,化成二阶常系数线性微分方程\\
	 
\end{flushleft}
\end{document}