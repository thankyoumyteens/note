\documentclass{article}
\usepackage{fontspec}
\usepackage{type1cm}
\usepackage{geometry}
\usepackage[bold-style=ISO]{unicode-math}
\usepackage[heading=true]{ctex}%添加heading=true,使用中文版式
\geometry{a4paper,left=1cm,right=1cm,top=3cm,bottom=3cm}
\usepackage{titlesec} %自定义多级标题格式的宏包
\titleformat{\section}[block]{\Huge\bfseries}{\arabic{section}}{1em}{}[]
\titleformat{\subsection}[block]{\huge\bfseries}{\arabic{section}.\arabic{subsection}}{1em}{}[]
\titleformat{\paragraph}[block]{\LARGE\bfseries}{[\arabic{paragraph}]}{1em}{}[]

\begin{document}
\begin{flushleft}
	\LARGE
	
	\section{公式}
	
	\subsection{微分方程的基本概念}
	
	\paragraph{微分方程}
	表示未知函数,未知函数的导数与自变量之间的关系的方程\\
	\paragraph{微分方程的阶}
	未知函数最高阶导数的阶数\\
	\paragraph{微分方程的解}
	找到一个函数代入微分方程能使该方程成为恒等式,找到的这个函数就是微分方程的解\\
	\paragraph{微分方程的通解}
	找到的微分方程中含有任意常数,即我们经常用C表示常数,这一类函数能使微分方程成为恒等式,统称为微分方程的通解\\
	\paragraph{微分方程的特解}
	在微分方程的基础上给出了初始条件(通常给出x和y的值关系)来确定出那个常数C,从而确定出一个函数,这个函数即为该微分方程的特解\\
	
	\subsection{一阶微分方程}
	
	
\end{flushleft}
\end{document}