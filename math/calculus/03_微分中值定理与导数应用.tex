\documentclass{article}
\usepackage{fontspec}
\usepackage{graphicx}
\usepackage{type1cm}
\usepackage{geometry}
\usepackage[bold-style=ISO]{unicode-math}
\usepackage[heading=true]{ctex}%添加heading=true,使用中文版式
\geometry{a4paper,left=1cm,right=1cm,top=3cm,bottom=3cm}
\usepackage{titlesec} %自定义多级标题格式的宏包
\titleformat{\section}[block]{\Huge\bfseries}{\arabic{section}}{1em}{}[]
\titleformat{\subsection}[block]{\huge\bfseries}{\arabic{section}.\arabic{subsection}}{1em}{}[]
\titleformat{\paragraph}[block]{\LARGE\bfseries}{[\arabic{paragraph}]}{1em}{}[]

\begin{document}
\begin{flushleft}
	\LARGE
	
	\section{公式}
	
	\subsection{微分中值定理}
	
	\paragraph{费马引理}
	$f(x)$在$U(x_0,\delta)$有定义,并在$x_0$可导,如果$f(x_0)$是极大(小)值,则$f'(x)=0$\\
	~\\
	\paragraph{罗尔定理}
	如果$f(x)$满足:\\
	1、在$[a,b]$上连续\\
	2、在$(a,b)$内可导\\
	3、$f(a)=f(b)$或$f(a)=f(c), a<c<b$\\
	则在$(a,b)$内至少有一点$\xi$,使$f'(\xi)=0$\\
	~\\
	若$f(x)$在$(a,b)$内有二阶导数,且$f(x_1)=f(x_2)=f(x_3)$,其中$a<x_1<x_2<x_3<b$,则在$(x_1,x_3)$内至少有一点$\xi$,使$f''(\xi)=0$\\
	推论:\\
	1、两个点相等,则二阶导为0\\
	2、三个点相等,则三阶导为0\\
	3、......\\
	~\\
	\paragraph{拉格朗日中值定理}
	如果$f(x)$满足:\\
	1、在$[a,b]$上连续\\
	2、在$(a,b)$内可导\\
	则在$(a,b)$内至少有一点$\xi$,使$\frac{f(b)-f(a)}{b-a}=f'(\xi)$\\
	~\\
	若函数$f(x)$在区间$I$上连续,在$I$内可导且导数恒为0,则$f(x)$在$I$上是一个常数\\
	~\\
	当$x>0$时,$\frac{x}{1+x}<ln(1+x)<x$\\
	~\\
	\paragraph{柯西中值定理}
	若$f(x)$和$F(x)$满足:\\
	1、在$[a,b]$上连续\\
	2、在$(a,b)$内可导\\
	3、对$\forall x\in (a,b),F'(x)\neq 0$\\
	则在$(a,b)$内至少有一点$\xi$,使$\frac{f(b-f(a)}{F(b)-F(a)}=\frac{f'(\xi)}{F'(\xi)}$\\
	
	\subsection{洛必达法则}
	
	若满足:\\
	1、求$\frac{0}{0}$型或$\frac{\infty}{\infty}$型的极限\\
	2、$\lim\limits_{x\to a}\frac{f'(x)}{F'(x)}$存在或为$\infty$\\
	则$\lim\limits_{x\to a}\frac{f(x)}{F(x)}=\lim\limits_{x\to a}\frac{f'(x)}{F'(x)}$\\
	
	\subsection{重要的等价}
	
	1、当$x\to 0$时,$x-\sin x \sim \frac{1}{6}x^3$\\
	2、\ \ $\sin(\arcsin x)=x$\\
	3、当$x\to 0$时,$\arcsin x-x \sim \frac{1}{6}x^3$\\
	4、当$x\to 0$时,$\tan x-x \sim \frac{1}{3}x^3$\\
	5、当$x\to 0$时,$x-\arctan x \sim \frac{1}{3}x^3$\\
	~\\
	6、当$x\to +\infty$时,$lnx << x^n$\\
	7、当$x\to +\infty$时,$x^n << e{\lambda x},(\lambda,n>0)$\\
	8、当$x\to +\infty$时,对数函数$<<$幂函数$<<$指数函数\\
	
	\subsection{泰勒公式}
	
	\paragraph{泰勒公式1}
	如果$f(x)$在$x_0$处有$n$阶导数,则对$\forall x \in U(x_0,\delta)$,有$f(x)=f(x_0)+f'(x_0)(x-x_0)+\frac{f''(x_0)}{2!}(x-x_0)^2+...+\frac{f^{(n)}(x_0)}{n!}(x-x_0)^n+o[(x-x_0)^n]$\\
	* 其中$o[(x-x_0)^n]$叫做佩亚诺余项\\
	~\\
	\paragraph{泰勒公式2}
	如果$f(x)$在$U(x_0,\delta)$内有$n+1$阶导数,则对$\forall x \in U(x_0,\delta)$,有$f(x)=f(x_0)+f'(x_0)(x-x_0)+\frac{f''(x_0)}{2!}(x-x_0)^2+...+\frac{f^{(n)}(x_0)}{n!}(x-x_0)^n+\frac{f^{(n+1)}(\xi)}{(n+1)!}(x-x_0)^{n+1}$\\
	* 其中$\frac{f^{(n+1)}(\xi)}{(n+1)!}(x-x_0)^{n+1}$叫做拉格朗日余项\\
	~\\
	若$x_0=0$,则上述泰勒公式又叫麦克劳林公式\\
	
	\subsection{重要的麦克劳林公式}
	
	1、$e^x=1+x+\frac{1}{2!}x^2+\frac{1}{3!}x^3+o(x^3)$\\
	2、$\sin x=x-\frac{1}{3!}x^3+o(x^3)$\\
	3、$\cos x=1-\frac{1}{2!}x^2+\frac{1}{4!}x^4+o(x^4)$\\
	4、$ln(1+x)=x-\frac{1}{2}x^2+\frac{1}{3}x^3+o(x^3)$\\
	5、$(1+x)^a=1+ax+\frac{a(a-1)}{2!}x^2+o(x^2)$\\
	
	\subsection{导数与函数的单调性}
	
	一阶导大于0,则函数单调增\\
	一阶导小于0,则函数单调减\\
	~\\
	驻点:导数为0的点\\
	~\\
	只有驻点和不可导的点才能成为单调区间的分界点\\
	
	\subsection{导数与曲线的凹凸性}
	
	二阶导大于0,则曲线是凹的\\
	二阶导小于0,则曲线是凸的\\
	~\\
	拐点:连续曲线凹与凸的分界点\\
	~\\
	拐点的二阶导为0或不存在\\
	~\\
	\paragraph{拐点的第一判别法}
	若在$\mathring{U}(x_0)$内二阶可导,则:\\
	1、$f''(x)$在$x_0$两侧变号,则$(x_0,f(x_0))$是拐点\\
	2、$f''(x)$在$x_0$两侧不变号,则$(x_0,f(x_0))$不是拐点\\
	~\\
	\paragraph{拐点的第二判别法}
	若$f''(x_0)=0$,则:\\
	1、$f^{'''}(x_0)\neq 0$,则$(x_0,f(x_0))$是拐点\\
	2、$f^{'''}(x_0)=0$,则没有结论\\
	~\\
	凹曲线的切线在曲线下面\\
	凸曲线的切线在曲线上面\\
	
	\subsection{导数与极值}
	
	极值点的一阶导为0或不存在\\
	~\\
	\paragraph{极值点的第一判别法}
	1、$f'(x)$在$x_0$两侧变号,则$(x_0,f(x_0))$是极值点\\
	\ \ $f'(x)$由正变负,则$f(x_0)$是极大值\\
	\ \ $f'(x)$由负变正,则$f(x_0)$是极小值\\
	2、$f'(x)$在$x_0$两侧不变号,则$(x_0,f(x_0))$不是极值点\\
	~\\
	\paragraph{极值点的第二判别法}
	若$f'(x_0)=0$,则:\\
	1、$f''(x_0)\neq 0$,则$(x_0,f(x_0))$是极值点\\
	\ \ $f''(x_0)>0$,则$f(x_0)$是极小值\\
	\ \ $f''(x_0)<0$,则$f(x_0)$是极大值\\
	2、$f''(x_0)=0$,则没有结论\\
	~\\
	若$f'(x_0)$到$f^{(n-1)}(x_0)$都为0,且$f^{(n)}\neq 0$,则:\\
	1、若$n$为奇数,则$f(x_0)$不是极值\\
	2、若$n$为偶数,则$f(x_0)$是极值\\
	\ \ $f^{(n)}(x_0)>0$,则$f(x_0)$是极小值\\
	\ \ $f^{(n)}(x_0)<0$,则$f(x_0)$是极大值\\
	
	\subsection{导数与最值}
	
	连续函数在闭区间内必由最值\\
	~\\
	求最值:\\
	1、求出$f(x)$在$(a,b)$内的所有驻点和不可导的点\\
	2、计算$f(x)$在驻点,不可导的点和端点a和b处的函数值\\
	3、比较这些函数值,最大的为最大值,最小的为最小值\\
	~\\
	若连续函数$f(x)$在$(a,b)$内有唯一的极值点$x_0$,则这个点就是最值点\\
	
	\subsection{渐近线}
	
	$x=a$是铅直渐近线$\Leftrightarrow \lim\limits_{x\to a^+}f(x)=\infty$或$\lim\limits_{x\to a^-}f(x)=\infty$\\
	~\\
	当$x\to +\infty$时,$y=b$是水平渐近线$\Leftrightarrow \lim\limits_{x\to +\infty}f(x)=b$\\
	~\\
	当$x\to +\infty$时,$y=kx+b$是斜渐近线$\Leftrightarrow \lim\limits_{x\to +\infty}\frac{f(x)}{x}=k\neq 0$,且$\lim\limits_{x\to +\infty}[f(x)-kx]=b$\\
	
	\subsection{曲率}
	
	若曲线由直角坐标方程$y=y(x)$给出,则曲率$k=\frac{|y''|}{(1+y^{'2})^{\frac{3}{2}}}$\\
	~\\
	若曲线由参数方程$\left\{
	\begin{array}{rcl}
	x=x(t)\\
	y=y(t)
	\end{array} \right.$给出,则曲率$k=\frac{|y''x'-y'x''|}{(x^{'2}+y^{'2})^{\frac{3}{2}}}$\\
	~\\
	曲率半径$R=\frac{1}{K}$\\
	
\end{flushleft}
\end{document}