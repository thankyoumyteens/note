\documentclass{article}
\usepackage{fontspec}
\usepackage{graphicx}
\usepackage{type1cm}
\usepackage{geometry}
\usepackage[bold-style=ISO]{unicode-math}
\usepackage[heading=true]{ctex}%添加heading=true,使用中文版式
\geometry{a4paper,left=1cm,right=1cm,top=3cm,bottom=3cm}
\usepackage{titlesec} %自定义多级标题格式的宏包
\titleformat{\section}[block]{\Huge\bfseries}{\arabic{section}}{1em}{}[]
\titleformat{\subsection}[block]{\huge\bfseries}{\arabic{section}.\arabic{subsection}}{1em}{}[]
\titleformat{\paragraph}[block]{\LARGE\bfseries}{[\arabic{paragraph}]}{1em}{}[]

\begin{document}
\begin{flushleft}
	\LARGE
	
	\section{高中基础}
	
	\subsection{根式有理化}
	若分母(或分子)是两个无理数相加(或相减),则把分子和分母同乘这两个无理数的和(或差),分母(或分子)就变成了有理数\\
	例:\\
	\qquad $\sqrt{x^2}+1-x=\frac{\sqrt{x^2}+1-x}{1}$\\
	\qquad $=\frac{(\sqrt{x^2}+1-x)(\sqrt{x^2}+1+x)}{\sqrt{x^2}+1+x}$\\
	\qquad $=\frac{x^2+1-x^2}{\sqrt{x^2}+1+x}=\frac{1}{\sqrt{x^2}+1+x}$\\
	
	\subsection{立方差公式}
	$a^3-b^3=(a-b)(a^2+ab+b^2)$\\
	例:\\
	\qquad $\frac{1}{1-x}-\frac{3}{1-x^3}$\\
	\qquad $=\frac{1+x+x^2}{(1-x)(1+x+x^2)}-\frac{3}{1-x^3}$\\
	\qquad $=\frac{1+x+x^2-3}{1-x^3}$\\
	\qquad $=\frac{x+x^2-2}{1-x^3}$\\
	
	\subsection{因式分解}
	1、$x^2+(a+b)x+ab=(x+a)(x+b)$\\
	例:\\
	\qquad $x^2-x-2$\\
	\qquad $=x^2+(2-1)x+(2\times(-1))=(x+2)(x-1)$\\
	
	\subsection{不等式}
	1、$||a|-|b||\le |a-b|$\\
	2、$|a\pm b|\le |a|+|b|$\\
	
	\subsection{对数函数}
	1、$log_a(M\times N)=log_aM+log_aN$\\
	2、$log_a\frac{M}{N}=log_aM-log_aN$\\
	3、$log_aM^N=Nlog_aM$\\
	4、$log_{a^N}M=\frac{1}{N}log_aM$\\
	5、$ln1=0$\\
	6、$lne=1$\\
	7、$M^N=e^{ln(M^N)}=e^{NlnM}$\\
	
	\subsection{三角函数}
	1、倍角公式:$sin2α=2sinαcosα$\\
	
	
	
	\section{函数}
	
	\subsection{取整函数}
	$y=[x]$ 向左取整: $x-1<[x]\leq x$\\
	一般搭配夹逼准则\\
	
	\subsection{分段函数}
	
	\paragraph{例题}
	设$f(x)=\left\{
	\begin{array}{rcl}
		1,& & |x|<1\\
		0,& & |x|=1\\
		-1,& & |x|>1
	\end{array} \right.,g(x)=e^x$,\\
	求$f[g(x)]$和$g[f(x)]$
	\paragraph{解}
	所有的$x$都换成$g(x)$:\\
	$f[g(x)]=\left\{
	\begin{array}{rcl}
		1,& & |e^x|<1,\mbox{即}x<0\\
		0,& & |e^x|=1,\mbox{即}x=0\\
		-1,& & |e^x|>1,\mbox{即}x>0
	\end{array} \right.$\\
	所有的$x$都换成$f(x)$:\\
	$g[f(x)]=e^{f(x)}=\left\{
	\begin{array}{rcl}
		e,& & |x|<1\\
		1,& & |x|=1\\
		e^{-1},& & |x|>1
	\end{array} \right.$\\
	
	\subsection{奇偶性}
	若$\forall x\in D$,有$f(-x)=-f(x)$,则$f(x)$为奇函数\\
	若$\forall x\in D$,有$f(-x)=f(x)$,则$f(x)$为偶函数\\
	
	\subsection{单调性}
	若$\forall x_1,x_2\in D$且$x_1<x_2$,有$f(x_1)<f(x_2)$,则$f(x)$在$D$上单调递增\\
	若$\forall x_1,x_2\in D$且$x_1<x_2$,有$f(x_1)>f(x_2)$,则$f(x)$在$D$上单调递减\\
	
	\subsection{有界性}
	若$\exists M>0$,对$\forall x\in D$,有$|f(x)|\le M$,则$f(x)$有界\\
	\qquad 若$\forall x\in D$,有$f(x)\ge M_1$,则$f(x)$有下界\\
	\qquad 若$\forall x\in D$,有$f(x)\le M_2$,则$f(x)$有上界\\
	
	\paragraph{例题}
	$y=xcosx$在$(-\infty,+\infty)$是否有界?\\
	是否为$x\to +\infty$的无穷大?
	\paragraph{解}
	取$x=2k\pi\in(-\infty,+\infty)$时,$y=2k\pi$大于任意的常数$M$,所以函数无界\\
	取$x=\frac{\pi}{2}+2k\pi\in(x,+\infty)$时,$y=0$,所以不是无穷大\\
	
	\subsection{周期性}
	若$\exists T>0$,对$\forall x\in D$且$x+T\in D$,有$f(x+T)=f(x)$,则$f(x)$有周期$T$\\
	
	
	\section{极限}
	
	\subsection{要分左右极限的情况}
	1、分段函数的分段点处\\
	2、e的无穷大型,如$\lim\limits_{x\to 1} e^{\frac{1}{x-1}}$\\
	3、$arctan\infty$型,如$\arctan{\frac{1}{x-1}}$\\
	
	\paragraph{例题}
	设$f(x)=\left\{
	\begin{array}{rcl}
		x-1,& & x<0\\
		0,& & x=0\\
		x+1,& & x>0
	\end{array} \right.$,\\
	证明当$x\to 0$时,$f(x)$的极限不存在
	\paragraph{解}
	$\lim\limits_{x\to 0^-}f(x)=x-1=-1$\\
	$\lim\limits_{x\to 0^+}f(x)=x+1=1$\\
	由于左右极限不相等,所以极限不存在\\
	
	\subsection{极限的四则运算}
	设$\lim f(x)=A,\lim g(x)=B$,则:\\
	1、$\lim [f(x)\pm g(x)]=A\pm B$\\
	2、$\lim [f(x)g(x)]=AB$\\
	3、$\lim \frac{f(x)}{g(x)} =\frac{A}{B},(B\neq 0)$\\
	~\\
	若$\lim f(x)$存在$\lim g(x)$不存在,则$\lim [f(x)\pm g(x)]$不存在,其他情况都没有结论\\
	~\\
	\subsection{多项式除多项式求极限}
	$\lim\limits_{x\to \infty} \frac{a_0x^m+...+a_mx^0}{b_0x^n+...+b_nx^0}=
	\left\{
	\begin{array}{rcl}
	\frac{a_0}{b_0},& & {m=n}\\
	0,& & {m<n}\\
	\infty,& & {m>n}
	\end{array} \right.$\\
	例:\\
	$\lim\limits_{x\to \infty} \frac{3x^3+4x^2+2}{7x^3+5x^2-3} = \frac{3}{7}$\\
	~\\
	\subsection{复合函数求极限}
	如果$f(x)$连续,且$g(x)$有极限A,则:\\
	$\lim\limits_{x\to x_0} f[g(x)]=f[\lim\limits_{x\to x_0}g(x)]=f(A)$\\
	
	\paragraph{例题}
	求极限$\lim\limits_{x\to 3}\sqrt{\frac{x-3}{x^2+9}}$
	\paragraph{解}
	$\lim\limits_{x\to 3}\sqrt{\frac{x-3}{x^2+9}}$\\
	$f=\sqrt{u}$是连续函数,且$u=\frac{x-3}{x^2+9}$有极限,则\\
	$=\sqrt{\lim\limits_{x\to 3}\frac{x-3}{x^2+9}}$\\
	$=\sqrt{\frac{1}{6}}$\\
	
	\subsection{幂指函数求极限}
	若$\lim f(x)=A>0$且$\lim g(x)=B$,则:$\lim f(x)^{g(x)}=A^B$\\
	
	\subsection{夹逼准则}
	函数$A>B>C$,函数$A$的极限是$X$,函数$C$的极限也是$X$,那么函数$B$的极限就一定是$X$\\
	
	\paragraph{例题}
	求极限$\lim\limits_{x\to 0^+}x[\frac{1}{x}]$
	\paragraph{解}
	使用取整函数的性质$x-1<[x]\le x$\\
	$\Rightarrow \frac{1}{x}-1<[\frac{1}{x}]\le \frac{1}{x}$\\
	由于$x\to 0^+$,知$x>0$\\
	$\Rightarrow 1-x<x[\frac{1}{x}]\le 1$\\
	由于$\lim\limits_{x\to 0^+}(1-x)=1$,且$\lim\limits_{x\to 0^+}1=1$\\
	由夹逼准则可得\\
	$\lim\limits_{x\to 0^+}x[\frac{1}{x}]=1$\\
	
	\subsection{单调有界准则}
	单调递增且有上界,则有极限,单调递减且有下界,则有极限\\
	
	\paragraph{例题}
	证明数列$\sqrt{2}, \sqrt{2+\sqrt{2}}, \sqrt{2+\sqrt{2+\sqrt{2}}}$有极限,并求极限
	\paragraph{解}
	设$x_{n+1}=\sqrt{2+x_n}$,$x_1=\sqrt{2}$\\
	由$x_1<\sqrt{2}<2$,设$x_k<2$,则$x_{k+1}=\sqrt{2+x_k}<\sqrt{2+2}=2$\\
	可知$x_n<2$,即数列有上界$2$\\
	判断单调性:\\
	$x_{n+1}-x_n$\\
	$=\sqrt{2+x_n}-x_n$\\
	根式有理化\\
	$=\frac{x_n+2-x_n^2}{\sqrt{x_n+2}+x_n}$\\
	$=\frac{(2-x_n)(1+x_n)}{\sqrt{x_n+2}+x_n}$\\
	由于$1+x_n, \sqrt{x_n+2}, x_n$都大于0,此时整个式子的正负由$2-x_n$决定\\
	由于$x_n$的上界为$2$,则$2-x_n>0$\\
	$x_{n+1}-x_n>0$即$x_{n+1}>x_n$\\
	则数列单调递增\\
	由单调有界准则知,数列有极限\\
	~\\
	令$\lim\limits_{x\to\infty}x_n=A$\\
	在$x_{n+1}=\sqrt{2+x_n}$两端取极限\\
	$\lim\limits_{x\to\infty}x_{n+1}=\lim\limits_{x\to\infty}\sqrt{2+x_n}$\\
	$\lim\limits_{x\to\infty}x_{n+1}=\sqrt{2+A}$\\
	$x_{n+1}$可以看作是$x_n$的子列\\
	子列与数列极限相同,则$\lim\limits_{x\to\infty}x_{n+1}=A$\\
	即$A=\sqrt{2+A}$\\
	$A=2$或$A=-1$\\
	由于数列大于0,则$A=2$\\
	
	\subsection{重要极限}
	1、$\lim\limits_{x\to 0} \frac{\sin x}{x}=1$\\
	\qquad $\lim\limits_{x\to 0} \frac{\tan x}{x}=1$\\
	\qquad $\lim\limits_{x\to 0} \frac{\arcsin x}{x}=1$\\
	1、推广$\lim\limits_{A\to 0} \frac{\sin A}{A}=1$,其中$A$为任意的表达式\\
	2、$\lim\limits_{x\to \infty} (1+\frac{1}{x})^x=e$\\
	\qquad $\lim\limits_{x\to 0} (1+x)^{\frac{1}{x}}=e$\\
	2、推广:$\lim (1+A)^{B}=e$,其中$A$和$B$为任意的表达式,且$A\to 0$,$B\to\infty$\\
	
	\section{无穷小}
	
	无穷小:极限为0的函数,(0也是无穷小)\\
	有界函数$\times$无穷小 仍是无穷小\\
	~\\
	设$\alpha$和$\beta$是无穷小,且$\alpha \neq 0$,
	若$\lim \frac{\beta}{\alpha}=0$,则$\beta$是比$\alpha$的高阶无穷小,
	记为:$\beta = o(\alpha)$\\
	~\\
	$o(x^2)\pm o(x^2)=o(x^2)$\\
	$o(x^2)\pm o(x^3)=o(x^2)$\\
	$x^2 o(x^3)=o(x^5)$\\
	$o(x^2) o(x^3)=o(x^5)$\\
	$o(2x^2)=o(x^2)$\\
	
	\paragraph{例题}
	求极限$\lim\limits_{x\to \infty}\frac{sinx}{x}$
	\paragraph{解}
	$\lim\limits_{x\to \infty}\frac{sinx}{x}$\\
	$=\lim\limits_{x\to \infty}sinx\frac{1}{x}$\\
	有界函数$\times$无穷小$=$无穷小\\
	$=0$\\
	
	\subsection{常用的等价}
	
	若$\lim \frac{\beta}{\alpha}=1$,则$\beta$与$\alpha$是等价无穷小,
	记为:$\beta \sim \alpha$\\
	~\\
	$\beta \sim \alpha \Leftrightarrow \beta = \alpha + o(\alpha)$\\
	$x$的高次方$\pm x$的低次方$\sim x$的低次方\\
	\qquad 例:$x^3+3x\sim 3x$\\
	~\\
	若$\alpha \sim \alpha_1$且$\beta \sim \beta_1$,则$\lim \frac{\beta}{\alpha} = \lim \frac{\beta_1}{\alpha_1}$\\
	~\\
	当$x\to 0$时,$\sin x \sim x$\\
	当$x\to 0$时,$\arcsin x \sim x$\\
	当$x\to 0$时,$\tan x \sim x$\\
	当$x\to 0$时,$\arctan x \sim x$\\
	~\\
	当$x\to 0$时,$ln(1+x) \sim x$\\
	当$x\to 0$时,$e^x-1 \sim x$\\
	~\\
	当$x\to 0$时,$1-\cos x \sim \frac{1}{2} x^2$\\
	当$x\to 0$时,$\sec x - 1 \sim \frac{1}{2} x^2$\\
	~\\
	当$x\to 0$时,$(1+\alpha x)^\beta -1 \sim \alpha\beta x$\\
	~\\
	当$x\to 0$时,$\alpha^x -1 \sim xln\alpha$\\
	
	\section{连续}
	
	若$f(x)$在$x_0$处连续,则$\lim\limits_{x\to x_0} f(x)=f(x_0)$\\
	~\\
	连续$\pm\times\div$连续$=$连续\\
	连续$\pm$不连续$=$不连续\\
	若$f(x)$连续$g(x)$也连续,则$f[g(x)]$连续\\
	~\\
	单调连续函数的反函数也连续,且单调性相同\\
	~\\
	初等函数在其定义域内都连续\\
	~\\
	闭区间内连续函数必有界\\
	推广:$f(x)$在$(a,b)$内连续,且$\lim\limits_{x\to a^+} f(x)$和$\lim\limits_{x\to b^-} f(x)$都存在,则$f(x)$在$(a,b)$内有界\\
	
	\paragraph{例题}
	求极限$\lim\limits_{x\to 2}\frac{x^3-1}{x^2-5x+3}$
	\paragraph{解}
	函数连续,则函数值与极限相等,直接代入极限\\
	$\lim\limits_{x\to 2}\frac{x^3-1}{x^2-5x+\sqrt3}=\frac{2^3-1}{2^2-5\times2+3}=\frac{7}{-3}$\\
	
	\subsection{零点定理}
	$f(x)$在$(a,b)$内连续,且$\lim\limits_{x\to a^+} f(x)$和$\lim\limits_{x\to b^-} f(x)$异号,则$\exists \xi \in (a,b)$,使得$f(\xi)=0$\\
	
	\subsection{间断点}
	
	\paragraph{第一类间断点}
	1、可去间断点:左右极限均存在且相等\\
	2、跳跃间断点:左右极限均存在且不相等\\
	\paragraph{第二类间断点}
	左右极限至少一个不存在\\
	1、无穷间断点:$x\to x_0^-$或$x\to x_0^+$时,$f(x)\to \infty$\\
	2、振荡间断点:$x\to x_0^-$或$x\to x_0^+$时,$f(x)$上下振荡\\
	
	\section{常用极限}
	
	1、当$a>1$时,$\lim\limits_{x\to-\infty}a^x=0$\\
	\qquad 例:$\lim\limits_{x\to 0^-}e^{\frac{1}{x}}=0$\\
	2、当$a>1$时,$\lim\limits_{x\to+\infty}a^x=+\infty$\\
	\qquad 例:$\lim\limits_{x\to 0^+}e^{\frac{1}{x}}=+\infty$\\
	3、$\lim\limits_{x\to\infty}\frac{\text{对数函数}}{\text{指数函数}}=0$\\
	\qquad 例:$\lim\limits_{x\to\infty}\frac{lnx}{x^a}=0$\\
	
	
	\section{各种类型的极限}
	
	\paragraph{0比0型}
	求极限$\lim\limits_{x\to 3}\frac{x-3}{x^2-9}$
	\paragraph{解}
	代入极限发现这是$\frac{0}{0}$型的极限\\
	则需要先消去$0$因子,再代入极限\\
	$\lim\limits_{x\to 3}\frac{x-3}{x^2-9}=\lim\limits_{x\to 3}\frac{x-3}{(x+3)(x-3)}=\lim\limits_{x\to 3}\frac{1}{x+3}=\frac{1}{3+3}=\frac{1}{6}$\\
	

	\paragraph{无穷比无穷型}
	求极限$\lim\limits_{x\to \infty}\frac{3x^3+4x^2+2}{7x^3+5x^2-3}$
	\paragraph{解}
	代入极限发现这是$\frac{\infty}{\infty}$型的极限\\
	则需要先消去$\infty$因子,再代入极限\\
	$\lim\limits_{x\to \infty}\frac{3x^3+4x^2+2}{7x^3+5x^2-3}$\\
	分子分母同除以最高次方\\
	$=\lim\limits_{x\to \infty}\frac{3+\frac{4}{x}+\frac{2}{x^3}}{7+\frac{5}{x}-\frac{3}{x^3}}$\\
	代入极限\\
	$=\frac{3+0+0}{7+0-0}=\frac{3}{7}$\\
	由此可推出多项式除多项式求极限的公式\\
	
	\paragraph{0乘无穷型}
	求极限$\lim\limits_{x\to \infty}x(\sqrt{x^2+1}-x)$
	\paragraph{解}
	代入极限发现这是$0\cdot\infty$型的极限\\
	则需要先化成$\frac{0}{0}$或$\frac{\infty}{\infty}$型的极限,再继续求\\
	$\lim\limits_{x\to \infty}x(\sqrt{x^2+1}-x)$\\
	根式有理化\\
	$=\lim\limits_{x\to \infty}\frac{x}{\sqrt{x^2+1}+x}$\\
	此时已变成$\frac{\infty}{\infty}$型的极限\\
	在$x\to\infty$时,$x^2+1$中的常数$1$对整体$x^2+1$的影响微乎其微,所以常数$1$可以忽略\\
	$=\lim\limits_{x\to \infty}\frac{x}{\sqrt{x^2}+x}
	=\lim\limits_{x\to \infty}\frac{x}{x+x}=\frac{1}{2}$\\
	
	\paragraph{无穷减无穷型}
	求极限$\lim\limits_{x\to 1}(\frac{1}{1-x}-\frac{3}{1-x^3})$
	\paragraph{解}
	代入极限发现这是$\infty-\infty$型的极限\\
	则需要先化成$\frac{0}{0}$或$\frac{\infty}{\infty}$型的极限,再继续求\\
	$\lim\limits_{x\to 1}(\frac{1}{1-x}-\frac{3}{1-x^3})$\\
	使用立方差公式化简\\
	$=\lim\limits_{x\to 1}\frac{x+x^2-2}{1-x^3}$\\
	此时已变成$\frac{0}{0}$型的极限\\
	因式分解\\
	$=\lim\limits_{x\to 1}\frac{(x+2)(x-1)}{(1-x)(1+x+x^2)}$\\
	$=-\frac{1+2}{1+1+1}=-1$\\
	
	\paragraph{1的无穷次方型}
	求极限$\lim\limits_{x\to 0}(1+2x)^{\frac{3}{sinx}}$
	\paragraph{解}
	代入极限发现这是$1^\infty$型的极限\\
	则需要凑重要极限2,即$(1+x)^{\frac{1}{x}}$\\
	$\lim\limits_{x\to 0}(1+2x)^{\frac{3}{sinx}}$\\
	$=\lim\limits_{x\to 0}(1+2x)^{\frac{1}{2x}\frac{6x}{sinx}}$\\
	幂指函数求极限\\
	$=\lim\limits_{x\to 0}(1+2x)^{\frac{1}{2x}\lim\limits_{x\to 0}\frac{6x}{sinx}}$\\
	$=e^6$\\

	\paragraph{0的0次方型}
	求极限$\lim\limits_{x\to 0^+}x^{sinx}$
	\paragraph{解}
	代入极限发现这是$0^0$型的极限\\
	则需要变形成$e^{ln}$的形式\\
	$=\lim\limits_{x\to 0^+}e^{sinx\cdot lnx}$\\
	此时变成了$0\cdot\infty$型的极限\\
	所以继续化成$\frac{0}{0}$或$\frac{\infty}{\infty}$型的极限\\
	将乘法变成除法:\\
	$=e^{\lim\limits_{x\to 0^+}\frac{lnx}{\frac{1}{sinx}}}$\\
	$=e^{\lim\limits_{x\to 0^+}\frac{lnx}{cscx}}$\\
	用洛必达法则\\
	$=e^{\lim\limits_{x\to 0^+}\frac{\frac{1}{x}}{-cscxcotx}}$\\
	$=e^{-\lim\limits_{x\to 0^+}\frac{sinxtanx}{x}}$\\
	$=e^{-\lim\limits_{x\to 0^+}\frac{x^2}{x}}$\\
	$=e^0=1$\\
	

\end{flushleft}
\end{document}