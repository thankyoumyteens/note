\documentclass{article}
\usepackage{fontspec}
\usepackage{graphicx}
\usepackage{type1cm}
\usepackage{geometry}
\geometry{a4paper,left=1cm,right=1cm,top=3cm,bottom=3cm}
\usepackage[bold-style=ISO]{unicode-math}
% 添加heading=true, 使用中文版式
\usepackage[heading=true]{ctex}
% 生成pdf书签
\usepackage{hyperref}
% 自定义多级标题格式的宏包
\usepackage{titlesec} 
\titleformat{\section}[block]{\Huge\bfseries}{\arabic{section}}{1em}{}[]
\titleformat{\subsection}[block]{\huge\bfseries}{\arabic{section}.\arabic{subsection}}{1em}{}[]
\titleformat{\paragraph}[block]{\LARGE\bfseries}{[\arabic{paragraph}]}{1em}{}[]

\begin{document}
\begin{flushleft}
	\LARGE
	
	\section{公式}
	
	\subsection{平面图形的面积}
	
	1、直角坐标\\
	\ \ $A=\int_{a}^{b}f(x)dx$\\
	\ \ $A=\int_{a}^{b}|f(x)-g(x)|dx$\\
	2、极坐标\\
	\ \ $A=\int_{\alpha}^{\beta}\frac{1}{2}r^2(\theta)d\theta$\\
	\ \ $A=\int_{\alpha}^{\beta}\frac{1}{2}[r_2^2(\theta)-r_1^2(\theta)]d\theta$\\
	
	\subsection{旋转体体积}
	
	1、绕x轴旋转\\
	\ \ $V=\int_{a}^{b}\pi f^2(x)dx$\\
	2、绕y轴旋转\\
	\ \ $V=\int_{a}^{b}2\pi xf(x)dx$\\
	
	\subsection{平面曲线的弧长}
	
	1、直角坐标\\
	\ \ $L=\int_{a}^{b}\sqrt{1+y^{'2}}dx$\\
	2、参数方程\\
	\ \ $L=\int_{\alpha}^{\beta}\sqrt{x_t^{'2}+y_t^{'2}}dt, (\alpha \le t \le \beta)$\\
	3、极坐标\\
	\ \ $A=\int_{\alpha}^{\beta}\sqrt{r^2(\theta)+r^{'2}(\theta)}d\theta$\\
	
	\subsection{重要公式}
	
	华里氏公式扩展\\
	$\int_{0}^{\pi}sin^nxdx=2\int_{0}^{\frac{\pi}{2}}sin^nxdx$\\
	$\int_{0}^{\pi}cos^nxdx=\left\{
	\begin{array}{rcl}
	0,& & n\mbox{为奇数}\\
	2\int_{0}^{\frac{\pi}{2}}cos^nxdx,& & n\mbox{为偶数}
	\end{array} \right.$\\
	$\int_{0}^{2\pi}sin^nxdx=\left\{
	\begin{array}{rcl}
	0,& & n\mbox{为奇数}\\
	4\int_{0}^{\frac{\pi}{2}}sin^nxdx,& & n\mbox{为偶数}
	\end{array} \right.$\\
	$\int_{0}^{2\pi}cos^nxdx=\left\{
	\begin{array}{rcl}
	0,& & n\mbox{为奇数}\\
	4\int_{0}^{\frac{\pi}{2}}sin^nxdx,& & n\mbox{为偶数}
	\end{array} \right.$\\
	~\\
	积分公式\\
	$\int \sqrt{x^2+a^2}dx=\frac{x}{2}\sqrt{x^2+a^2}+\frac{a^2}{2}ln(x+\sqrt{x^2+a^2})+C$\\
	$\int \sqrt{x^2-a^2}dx=\frac{x}{2}\sqrt{x^2-a^2}-\frac{a^2}{2}ln(x+\sqrt{x^2-a^2})+C$\\
	$\int \sqrt{a^2+x^2}dx=\frac{x}{2}\sqrt{a^2+x^2}+\frac{a^2}{2}arcsin\frac{x}{a}+C$\\
	
\end{flushleft}
\end{document}