\documentclass{article}
\usepackage{fontspec}
\usepackage{type1cm}
\usepackage{geometry}
\usepackage[bold-style=ISO]{unicode-math}
\usepackage[heading=true]{ctex}%添加heading=true,使用中文版式
\geometry{a4paper,left=1cm,right=1cm,top=3cm,bottom=3cm}
\usepackage{titlesec} %自定义多级标题格式的宏包
\titleformat{\section}[block]{\Huge\bfseries}{\arabic{section}}{1em}{}[]
\titleformat{\subsection}[block]{\huge\bfseries}{\arabic{section}.\arabic{subsection}}{1em}{}[]
\titleformat{\paragraph}[block]{\huge\bfseries}{[\arabic{paragraph}]}{1em}{}[]

\begin{document}
\begin{flushleft}
	\huge
	
	\section{公式}
	
	\subsection{不定积分}
	
	若$F^{'}(x)=f(x)$,则$F(x)$叫原函数\\
	~\\
	连续函数一定有原函数\\
	~\\
	不定积分:$\int f(x)dx=F(x)+C$\\
	
	\subsection{基本积分表}
	
	$\int kdx=kx+C$\\
	$\int x^udx=\frac{x^{u+1}}{u+1}+C$\\
	$\int \frac{1}{x}dx=ln|x|+C$\\
	$\int \frac{1}{1+x^2}dx=\arctan x+C$\\
	$\int \frac{1}{\sqrt{1-x^2}}=\arcsin x+C$\\
	$\int \cos xdx=\sin x+C$\\
	$\int \sin xdx=-\cos x+C$\\
	$\int \frac{1}{\cos^2x}=\tan x+C$\\
	$\int \sec^2x=\tan x+C$\\
	$\int \frac{1}{\sin^2x}=-\cot x+C$\\
	$\int \csc^2x=-\cot x+C$\\
	$\int \sec x\tan xdx=\sec x+C$\\
	$\int \csc x\cot xdx=-\csc x+C$\\
	$\int e^xdx=e^x+C$\\
	$\int a^xdx=\frac{a^x}{lna}+C$\\
	$\int \tan xdx=-ln|\cos x|+C$\\
	$\int \cot xdx=ln|\sin x|+C$\\
	$\int \sec xdx=ln|\sec x+\tan x|+C$\\
	$\int \csc xdx=ln|\csc x-\cot x|+C$\\
	$\int \frac{1}{a^2+x^2}dx=\frac{1}{a}\arctan \frac{x}{a}+C$\\
	$\int \frac{1}{x^2-a^2}dx=\frac{1}{2a}ln|\frac{x-a}{x+a}|+C$\\
	$\int \frac{1}{\sqrt{a^2-x^2}}dx=\arcsin \frac{x}{a}+C$\\
	$\int \frac{1}{\sqrt{x^2+a^2}}dx=ln(x+\sqrt{x^2+a^2})+C$\\
	$\int \frac{1}{\sqrt{x^2-a^2}}dx=ln|x+\sqrt{x^2-a^2}|+C$\\
	
	\subsection{不定积分的性质}
	
	$\frac{d}{dx}[\int f(x)dx]=f(x)$\\
	$\int F^{'}(x)dx=F(x)$\\
	$\int [f(x)\pm g(x)]dx=\int f(x)dx \pm \int g(x)dx$\\
	$\int kf(x)dx=k\int f(x)dx$\\
	~\\
	设$f(x)$在区间$I$上除$x=c$之外处处连续,且$x=c$是$f(x)$的第一类间断点,则$f(x)$在$I$上没有原函数\\
	~\\
	$f(x)$是奇函数,则$\int f(x)dx$和$f^{'}(x)$是偶函数\\
	
	\subsection{换元积分法}
	
	第一类换元法:设$\int f(x)dx=F(x)+C$,则$\int f[\phi(x)]\phi^{'}(x)dx=\int f[\phi(x)]d\phi(x)=F[\phi(x)]+C$\\
	~\\
	常用的换元:\\
	$\frac{1}{x}dx=dlnx$\\
	$\frac{1}{\sqrt{x}}dx=d2\sqrt{x}$\\
	$\frac{1}{x^2}dx=d(-\frac{1}{x})$\\
	~\\
	如果被积函数都是由$\sin$或$\cos$组成,则:\\
	1.若$f(-\sin x,\cos x)=-f(\sin x,\cos x)$,则凑$d\cos x$\\
	2.若$f(\sin x,-\cos x)=-f(\sin x,\cos x)$,则凑$d\sin x$\\
	3.若$f(-\sin x,-\cos x)=f(\sin x,\cos x)$,则凑$d\tan x$\\
	~\\
	第二类换元法:\\
	1.三角代换\\
	\ \ $\sqrt{a^2-x^2} \Rightarrow x=a\sin t, -\frac{\pi}{2}<t<\frac{\pi}{2}$\\
	\ \ $\sqrt{a^2+x^2} \Rightarrow x=a\tan t, -\frac{\pi}{2}<t<\frac{\pi}{2}$\\
	\ \ $\sqrt{x^2-a^2} \Rightarrow x=a\sec t, 0<t<\frac{\pi}{2}$\\
	2.根式代换\\
	\ \ $\sqrt[n]{ax+b} \Rightarrow t$\\
	\ \ $\sqrt[n]{\frac{ax+b}{cx+d}} \Rightarrow t$\\
	3.倒代换:若被积函数分母的次方比分子的次方高两次及以上,则代换为$x=\frac{1}{t}$\\
	
	\subsection{分部积分法}
	
	$\int udv=uv-\int vdu$\\
	~\\
	何时用?两类不同的函数相乘做积分\\
	1.$P_n(x)$作$u \left\{
	\begin{array}{lcl}
	\int P_n(x)e^{ax}dx\\
	\int P_n(x)\sin axdx\\
	\int P_n(x)\cos axdx
	\end{array} \right.$\\
	2.$P_n(x)$作$v \left\{
	\begin{array}{lcl}
	\int P_n(x)lnxdx\\
	\int P_n(x)\arcsin axdx\\
	\int P_n(x)\arctan axdx
	\end{array} \right.$\\
	3.均可作$u \left\{
	\begin{array}{lcl}
	\int e^{ax}\sin \beta xdx\\
	\int e^{ax}\cos \beta xdx
	\end{array} \right.$\\
	
	\subsection{有理函数积分法}
	
	前提:$\frac{P(x)}{Q(x)}$,$P$和$Q$是多项式,$Q(x)$可以因式分解\\
	1.若$Q(x)$中有一个因子$(x-a)^n$,则$\frac{P(x)}{Q(x)}$的分解式中有$\frac{A_1}{x-a}+\frac{A_2}{(x-a)^2}+...+\frac{A_n}{(x-a)^n}$\\
	2.若$Q(x)$中有一个因子$(x^2+px+q)$且$p^2-4q<0$,则$\frac{P(x)}{Q(x)}$的分解式中有$\frac{A_1x+B_1}{x^2+px+q}+\frac{A_2x+B_2}{(x^2+px+q)^2}+...+\frac{A_nx+B_n}{(x^2+px+q)^n}$\\
\end{flushleft}
\end{document}