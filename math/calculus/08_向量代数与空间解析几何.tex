\documentclass{article}
\usepackage{fontspec}
\usepackage{type1cm}
\usepackage{geometry}
\usepackage[bold-style=ISO]{unicode-math}
\usepackage[heading=true]{ctex}%添加heading=true,使用中文版式
\geometry{a4paper,left=1cm,right=1cm,top=3cm,bottom=3cm}
\usepackage{titlesec} %自定义多级标题格式的宏包
\titleformat{\section}[block]{\Huge\bfseries}{\arabic{section}}{1em}{}[]
\titleformat{\subsection}[block]{\huge\bfseries}{\arabic{section}.\arabic{subsection}}{1em}{}[]
\titleformat{\paragraph}[block]{\LARGE\bfseries}{[\arabic{paragraph}]}{1em}{}[]

\begin{document}
\begin{flushleft}
	\LARGE
	% \qquad 缩进两个字符
	
	\section{公式}

	\subsection{向量的模}
	
	设$\vec{a}=(x,y,z)$,则$|\vec{a}|=\sqrt{x^2+y^2+z^2}$\\
	~\\
	设$\alpha,\beta,\gamma$是向量$\vec{a}$与坐标轴的夹角,则$\vec{a}$的方向余弦分别为$cos\alpha=\frac{x}{\vec{a}},cos\beta=\frac{y}{\vec{a}},cos\gamma=\frac{z}{\vec{a}}$\\
	
	\subsection{向量的线性运算}
	
	1、$\vec{a}+\vec{b}=(a_x+b_x,a_y+b_y,a_z+b_z)$\\
	2、$\lambda\vec{a}=(\lambda a_x,\lambda a_y,\lambda a_z)$\\
	
	\subsection{向量的数量积}
	
	1、$\vec{a}\cdot\vec{b}=|\vec{a}||\vec{b}|cos\theta$,其中$\theta$是两个向量的夹角\\
	2、$\vec{a}\cdot\vec{b}=a_xb_x+a_yb_y+a_zb_z$\\
	3、两个向量垂直,即$\vec{a}\perp\vec{b}\Leftrightarrow\vec{a}\cdot\vec{b}=0$\\
	
	\subsection{向量的向量积}
	
	1、向量积的值:$|\vec{c}|=|\vec{a}\times\vec{b}|=|\vec{a}||\vec{b}|sin\theta$,其中$\theta$是两个向量的夹角\\
	2、向量积的方向:$\vec{c}$同时垂直于$\vec{a}$和$\vec{b}$\\
	3、行列式表示:$\vec{a}\times\vec{b}=
	\left|\begin{array}{cccc}
	\vec{i}&\vec{j}&\vec{k}\\
	a_x&a_y&a_z\\
	b_x&b_y&b_z
	\end{array}\right|$,其中$\vec{i},\vec{j},\vec{k}$是向量积在三个坐标轴上的分量\\
	4、两个向量平行,即$\vec{a}//\vec{b}\Leftrightarrow\vec{a}\times\vec{b}=0\Leftrightarrow\frac{a_x}{b_x}=\frac{a_y}{b_y}=\frac{a_z}{b_z}$\\
	
	\subsection{向量的混合积}
	
	1、$(\vec{a}\vec{b}\vec{c})=(\vec{a}\times\vec{b})\cdot\vec{c}=
	\left|\begin{array}{cccc}
	a_x&a_y&a_z\\
	b_x&b_y&b_z\\
	c_x&c_y&c_z
	\end{array}\right|$\\
	2、$\vec{a},\vec{b},\vec{c}$共面$\Leftrightarrow(\vec{a}\vec{b}\vec{c})=0$\\
	
	
	
\end{flushleft}
\end{document}